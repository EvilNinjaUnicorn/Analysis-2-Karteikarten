%*********************Preambel***********************


\usepackage[a6paper, landscape, top=20mm]{geometry}
\usepackage{fancyhdr}
\pagestyle{fancy}
\pagenumbering{gobble}
\newcommand{\subject}[1]{\rhead{#1}}

\newenvironment{flashcard}[2][\phantom{x}]{
  \newcommand{\backside}{\newpage}
  \pagestyle{fancy}
  \lhead{#2}
  \chead{#1}
}{
\pagestyle{empty} \pagenumbering{gobble}
  \newpage
}

\newcommand{\centerall}[1]{
    \vspace*{\fill}
    \begin{center}
      \begin{minipage}{.6\textwidth}
        #1
      \end{minipage}
    \end{center}
    \vfill
  }



\newenvironment{tabulator}[0]{
  \par
  \begingroup
  \leftskip=0.5cm
}{
    \par
    \endgroup
}


%Encodingeinstellung (Umlaute)
\usepackage[utf8x]{inputenc}
%Font-Encoding 
\usepackage[T1]{fontenc}

%Glaettung von Text
\usepackage{lmodern}


%Formatierung Seite
%\usepackage{geometry}
%\geometry{left=2cm,right=2cm,top=2.3cm,bottom=2.3cm,includeheadfoot}


%Spracheinstellung Deutsch
\usepackage[ngerman]{babel}
%Zeilenabstände
\usepackage{setspace}
%sin­glespac­ing
\onehalfspacing
%dou­blespac­ing
\addto\captionsngerman{
\renewcommand{\figurename}{Abb.}
\renewcommand{\tablename}{Tab.}
}


%%Spracheinstellung Englisch
%\usepackage[ngerman,english]{babel}
%\usepackage{setspace}
%sin­glespac­ing
%\onehalfspacing
%dou­blespac­ing
%\addto\captionsenglish{
%\renewcommand{\figurename}{Fig.}
%\renewcommand{\tablename}{Tab.}
%}


%Packete zur Formatierung von Tabellen und Grafiken
\usepackage{graphicx}
\usepackage{tabularx}
\usepackage{multicol}
\usepackage{float}
\usepackage{floatflt}
\usepackage{here}
\usepackage{blindtext}
\usepackage{wrapfig}
\usepackage{bigstrut}
\usepackage{subfloat}
\usepackage{subfigure}
\usepackage{multirow}
%Ueberschrift mit Serifen (nur Inhaltsverzeichnis)
%\setkomafont{sectioning}{\normalfont\bfseries}


%Caption 
\usepackage{caption}
%\captionsetup{font=small}


%Erlaubt Farblose Hyperrefs
\usepackage[breaklinks,pdfborder={0 0 0},bookmarksopen=true]{hyperref}
\usepackage{url}


%Mathepackete
%\usepackage{amsthm}
\usepackage{amsmath}
\usepackage{amsfonts}
\usepackage{amssymb}
\usepackage{textgreek}
\usepackage{xargs}
\usepackage{bbm}
\usepackage{mathtools}
\usepackage{ntheorem}


\theoremstyle{break}
\newtheorem*{definition}{Definition}
\theoremstyle{break}
\newtheorem*{satz}{Satz}
\theoremstyle{break}
\newtheorem*{lemma}{Lemma}
\theoremstyle{break}
\newtheorem*{korollar}{korollar}
\theoremstyle{break}
\newtheorem*{beispiel}{Beilspiel}

\newcommand{\with}{\textbf{ \textup{mit} \allowbreak}}
\newcommand{\then}{\textbf{ \textup{gilt} \allowbreak}}
\newcommandx{\limes}[2][1=n, 2=\infty]{\lim\limits_{#1 \to #2}}
\newcommandx{\summation}[2][1=n, 2={i=0}]{\sum\limits_{#2}^{#1}}
\newcommandx{\product}[2][1=n, 2={i=1}]{\prod\limits_{#2}^{#1}}
\newcommandx{\integral}[2][1=b, 2=a]{\int\limits_{#2}^{#1}}
\newcommand{\N}{\mathbb{N}}
\newcommand{\Z}{\mathbb{Z}}
\newcommand{\Q}{\mathbb{Q}}
\newcommand{\R}{\mathbb{R}}
\newcommand{\C}{\mathbb{C}}
\newcommand{\I}{\mathbb{I}}
\renewcommand{\L}{\mathbb{L}}
\newcommand{\F}{\mathbb{F}}


\newcommand{\rekursiveDefinition}[6][]{
\begin{definition} (rekursiv)\\
$\text{#2}#1 := \textbf{rekursiv } #3 \text{ in } #4 \text{ durch}$\\
   \phantom{x} $\quad #3 (1) := #5;$\\
   \phantom{x} $\quad #3 (n) := #6;$\\
 $\Box;$
\end{definition}
}


\newcommand{\mean}[1]{\bar{#1}}



% Abstand nach math-Umgebungen
\setlength\abovedisplayshortskip{0pt}
\setlength\belowdisplayshortskip{0pt}
\setlength\abovedisplayskip{0pt}
\setlength\belowdisplayskip{0pt}


%Vektoren
\renewcommand{\vec}[1]{\boldsymbol{#1}}


%Roemische Zahlen
\newcommand{\Rom}[1]{\MakeUppercase{\romannumeral #1}}
\newcommand{\rom}[1]{\romannumeral #1}


%Units und Fractioneinstellungen
%Intelligent Comma (Abstand von Komma)
\usepackage{icomma}
%Darstellung von SI Einheiten
\usepackage{siunitx}
\sisetup{per-mode=reciprocal}  %z.B. s^(-1)
%\sisetup{per-mode=reciprocal} %z.B. 1/s



%%Nummerierung Gleichungen bei mehreren Kapiteln
%\numberwithin{equation}{section}
%\numberwithin{figure}{section}
%\numberwithin{table}{section}


%Farben
\usepackage{color}
\definecolor{grau}{rgb}{0.95,0.95,0.95}
\definecolor{dunkelgrau}{rgb}{0.8,0.8,0.8}
%Ermoeglicht Tabellen zu faerben
\usepackage{colortbl}


%Keine Einrueckungen im Fliesstext (nach Umbruch)
\setlength{\parindent}{0pt}


%Formatierung Kopfzeile Typ 2
%\usepackage[automark,headsepline]{scrpage2}       % Kopf und Fusszeilen-Layout
%\renewcommand{\headfont}{\normalfont\sffamily}    % Kolumnentitel serifenlos
%\renewcommand{\pnumfont}{\normalfont\sffamily}    % Seitennummern serifenlos
%\pagestyle{scrheadings}
%\automark{chapter}
%\automark{section}
%\ohead{\pagemark}
%\ihead{\headmark}
%\cfoot{}     


%Ueberschriften formatieren
%\addtokomafont{title}{\normalfont\bfseries}
%\addtokomafont{section}{\normalfont\bfseries\Large}
%\addtokomafont{subsection}{\normalfont\bfseries\large}
%\addtokomafont{subsubsection}{\normalfont\bfseries\normalsize}
%\addtokomafont{paragraph}{\normalfont\bfseries\normalsize}


%%% Vermeidung von Hurenkindern und Schusterjungen
\clubpenalty = 10000  % schliesst Schusterjungen aus
\widowpenalty = 10000 % schliesst Hurenkinder aus


% Zeilenabstände in Tabellen
\renewcommand{\arraystretch}{1.4}

%%Neue Kommandos (hier ein Auszug aus meinen) 
%\newcommand{\del}{\partial}
%\renewcommand{\phi}{\varphi}
%\renewcommand{\epsilon}{\varepsilon}
%\newcommand{\cc}{^{\circ}}
%\newcommand{\cld}{\cellcolor{dunkelgrau}}
%\newcommand{\clg}{\cellcolor{grau}}
%\renewcommand{\si}[2]{\SI{#1}{#2}}
%\newcommand{\ba}[1]{\begin{align}#1\end{align}}
%\newcommand{\be}[1]{\begin{equation}#1\end{equation}}
%\renewcommand{\sectionmark}[1]{\markright{#1}}

%Definierte Wortsilbentrennung
\hyphenation{Test}


%Graphiken Zeichnen
\usepackage{tikz}
\usetikzlibrary{calc,patterns,decorations.pathmorphing, intersections}


\usepackage{multirow}
\usepackage{pdfpages}
\usepackage{pgfplots}
\usepackage{caption}
\captionsetup{format=plain}



%*********************Meine Preambel***********************


% %Stichwortverzeichnis
% \usepackage{makeidx}
% \makeindex
% 
% 
% %Numerierungstiefe
% \setcounter{secnumdepth}{3}
% \setcounter{tocdepth}{2}
% 
% %Verzeichnis verlinken
% \usepackage{hyperref}
% 
% 
% %Farben (Inhaltsverzeichnis)
% \usepackage{color}
% \definecolor{black}{rgb}{0,0,0}
% \hypersetup{colorlinks, linkcolor=black}
% 
% %Zitieren
% \usepackage{cite}
% 
% %Kopf und Fusszeilen
% \usepackage{fancyhdr}
% \pagestyle{fancy} %vordefinierter Style
% \fancyhf{}
% %Kopfzeile
% \fancyhead[L]{\nouppercase{\leftmark}}
% %\fancyhead[C]{TG 12/3}
% \fancyhead[R]{\today}
% \renewcommand{\headrulewidth}{0.5pt}
% \headheight 15pt
% %Fusszeile
% \fancyfoot[L]{Emanuel Hubenschmid}
% \fancyfoot[C]{\thepage}
% \fancyfoot[R]{}
% \renewcommand{\footrulewidth}{0.0pt}
% \footskip 63pt
% 
% %Tabellen
% 
% \usepackage{tabularx}
% \usepackage{float}
% \floatplacement{figure}{H}
% 
