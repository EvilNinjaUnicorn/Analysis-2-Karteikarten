\documentclass{article}

%*********************Preambel***********************


\usepackage[a6paper, landscape, top=20mm]{geometry}
\usepackage{fancyhdr}
\pagestyle{fancy}
\pagenumbering{gobble}
\newcommand{\subject}[1]{\rhead{#1}}

\newenvironment{flashcard}[2][\phantom{x}]{
  \newcommand{\backside}{\newpage}
  \pagestyle{fancy}
  \lhead{#2}
  \chead{#1}
}{
\pagestyle{empty} \pagenumbering{gobble}
  \newpage
}

\newcommand{\centerall}[1]{
    \vspace*{\fill}
    \begin{center}
      \begin{minipage}{.6\textwidth}
        #1
      \end{minipage}
    \end{center}
    \vfill
  }



\newenvironment{tabulator}[0]{
  \par
  \begingroup
  \leftskip=0.5cm
}{
    \par
    \endgroup
}


%Encodingeinstellung (Umlaute)
\usepackage[utf8x]{inputenc}
%Font-Encoding 
\usepackage[T1]{fontenc}

%Glaettung von Text
\usepackage{lmodern}


%Formatierung Seite
%\usepackage{geometry}
%\geometry{left=2cm,right=2cm,top=2.3cm,bottom=2.3cm,includeheadfoot}


%Spracheinstellung Deutsch
\usepackage[ngerman]{babel}
%Zeilenabstände
\usepackage{setspace}
%sin­glespac­ing
\onehalfspacing
%dou­blespac­ing
\addto\captionsngerman{
\renewcommand{\figurename}{Abb.}
\renewcommand{\tablename}{Tab.}
}


%%Spracheinstellung Englisch
%\usepackage[ngerman,english]{babel}
%\usepackage{setspace}
%sin­glespac­ing
%\onehalfspacing
%dou­blespac­ing
%\addto\captionsenglish{
%\renewcommand{\figurename}{Fig.}
%\renewcommand{\tablename}{Tab.}
%}


%Packete zur Formatierung von Tabellen und Grafiken
\usepackage{graphicx}
\usepackage{tabularx}
\usepackage{multicol}
\usepackage{float}
\usepackage{floatflt}
\usepackage{here}
\usepackage{blindtext}
\usepackage{wrapfig}
\usepackage{bigstrut}
\usepackage{subfloat}
\usepackage{subfigure}
\usepackage{multirow}
%Ueberschrift mit Serifen (nur Inhaltsverzeichnis)
%\setkomafont{sectioning}{\normalfont\bfseries}


%Caption 
\usepackage{caption}
%\captionsetup{font=small}


%Erlaubt Farblose Hyperrefs
\usepackage[breaklinks,pdfborder={0 0 0},bookmarksopen=true]{hyperref}
\usepackage{url}


%Mathepackete
%\usepackage{amsthm}
\usepackage{amsmath}
\usepackage{amsfonts}
\usepackage{amssymb}
\usepackage{textgreek}
\usepackage{xargs}
\usepackage{bbm}
\usepackage{mathtools}
\usepackage{ntheorem}


\theoremstyle{break}
\newtheorem*{definition}{Definition}
\theoremstyle{break}
\newtheorem*{satz}{Satz}
\theoremstyle{break}
\newtheorem*{lemma}{Lemma}
\theoremstyle{break}
\newtheorem*{korollar}{korollar}
\theoremstyle{break}
\newtheorem*{beispiel}{Beilspiel}

\newcommand{\with}{\textbf{ \textup{mit} \allowbreak}}
\newcommand{\then}{\textbf{ \textup{gilt} \allowbreak}}
\newcommandx{\limes}[2][1=n, 2=\infty]{\lim\limits_{#1 \to #2}}
\newcommandx{\summation}[2][1=n, 2={i=0}]{\sum\limits_{#2}^{#1}}
\newcommandx{\product}[2][1=n, 2={i=1}]{\prod\limits_{#2}^{#1}}
\newcommandx{\integral}[2][1=b, 2=a]{\int\limits_{#2}^{#1}}
\newcommand{\N}{\mathbb{N}}
\newcommand{\Z}{\mathbb{Z}}
\newcommand{\Q}{\mathbb{Q}}
\newcommand{\R}{\mathbb{R}}
\newcommand{\C}{\mathbb{C}}
\newcommand{\I}{\mathbb{I}}
\renewcommand{\L}{\mathbb{L}}
\newcommand{\F}{\mathbb{F}}


\newcommand{\rekursiveDefinition}[6][]{
\begin{definition} (rekursiv)\\
$\text{#2}#1 := \textbf{rekursiv } #3 \text{ in } #4 \text{ durch}$\\
   \phantom{x} $\quad #3 (1) := #5;$\\
   \phantom{x} $\quad #3 (n) := #6;$\\
 $\Box;$
\end{definition}
}


\newcommand{\mean}[1]{\bar{#1}}



% Abstand nach math-Umgebungen
\setlength\abovedisplayshortskip{0pt}
\setlength\belowdisplayshortskip{0pt}
\setlength\abovedisplayskip{0pt}
\setlength\belowdisplayskip{0pt}


%Vektoren
\renewcommand{\vec}[1]{\boldsymbol{#1}}


%Roemische Zahlen
\newcommand{\Rom}[1]{\MakeUppercase{\romannumeral #1}}
\newcommand{\rom}[1]{\romannumeral #1}


%Units und Fractioneinstellungen
%Intelligent Comma (Abstand von Komma)
\usepackage{icomma}
%Darstellung von SI Einheiten
\usepackage{siunitx}
\sisetup{per-mode=reciprocal}  %z.B. s^(-1)
%\sisetup{per-mode=reciprocal} %z.B. 1/s



%%Nummerierung Gleichungen bei mehreren Kapiteln
%\numberwithin{equation}{section}
%\numberwithin{figure}{section}
%\numberwithin{table}{section}


%Farben
\usepackage{color}
\definecolor{grau}{rgb}{0.95,0.95,0.95}
\definecolor{dunkelgrau}{rgb}{0.8,0.8,0.8}
%Ermoeglicht Tabellen zu faerben
\usepackage{colortbl}


%Keine Einrueckungen im Fliesstext (nach Umbruch)
\setlength{\parindent}{0pt}


%Formatierung Kopfzeile Typ 2
%\usepackage[automark,headsepline]{scrpage2}       % Kopf und Fusszeilen-Layout
%\renewcommand{\headfont}{\normalfont\sffamily}    % Kolumnentitel serifenlos
%\renewcommand{\pnumfont}{\normalfont\sffamily}    % Seitennummern serifenlos
%\pagestyle{scrheadings}
%\automark{chapter}
%\automark{section}
%\ohead{\pagemark}
%\ihead{\headmark}
%\cfoot{}     


%Ueberschriften formatieren
%\addtokomafont{title}{\normalfont\bfseries}
%\addtokomafont{section}{\normalfont\bfseries\Large}
%\addtokomafont{subsection}{\normalfont\bfseries\large}
%\addtokomafont{subsubsection}{\normalfont\bfseries\normalsize}
%\addtokomafont{paragraph}{\normalfont\bfseries\normalsize}


%%% Vermeidung von Hurenkindern und Schusterjungen
\clubpenalty = 10000  % schliesst Schusterjungen aus
\widowpenalty = 10000 % schliesst Hurenkinder aus


% Zeilenabstände in Tabellen
\renewcommand{\arraystretch}{1.4}

%%Neue Kommandos (hier ein Auszug aus meinen) 
%\newcommand{\del}{\partial}
%\renewcommand{\phi}{\varphi}
%\renewcommand{\epsilon}{\varepsilon}
%\newcommand{\cc}{^{\circ}}
%\newcommand{\cld}{\cellcolor{dunkelgrau}}
%\newcommand{\clg}{\cellcolor{grau}}
%\renewcommand{\si}[2]{\SI{#1}{#2}}
%\newcommand{\ba}[1]{\begin{align}#1\end{align}}
%\newcommand{\be}[1]{\begin{equation}#1\end{equation}}
%\renewcommand{\sectionmark}[1]{\markright{#1}}

%Definierte Wortsilbentrennung
\hyphenation{Test}


%Graphiken Zeichnen
\usepackage{tikz}
\usetikzlibrary{calc,patterns,decorations.pathmorphing, intersections}


\usepackage{multirow}
\usepackage{pdfpages}
\usepackage{pgfplots}
\usepackage{caption}
\captionsetup{format=plain}



%*********************Meine Preambel***********************


% %Stichwortverzeichnis
% \usepackage{makeidx}
% \makeindex
% 
% 
% %Numerierungstiefe
% \setcounter{secnumdepth}{3}
% \setcounter{tocdepth}{2}
% 
% %Verzeichnis verlinken
% \usepackage{hyperref}
% 
% 
% %Farben (Inhaltsverzeichnis)
% \usepackage{color}
% \definecolor{black}{rgb}{0,0,0}
% \hypersetup{colorlinks, linkcolor=black}
% 
% %Zitieren
% \usepackage{cite}
% 
% %Kopf und Fusszeilen
% \usepackage{fancyhdr}
% \pagestyle{fancy} %vordefinierter Style
% \fancyhf{}
% %Kopfzeile
% \fancyhead[L]{\nouppercase{\leftmark}}
% %\fancyhead[C]{TG 12/3}
% \fancyhead[R]{\today}
% \renewcommand{\headrulewidth}{0.5pt}
% \headheight 15pt
% %Fusszeile
% \fancyfoot[L]{Emanuel Hubenschmid}
% \fancyfoot[C]{\thepage}
% \fancyfoot[R]{}
% \renewcommand{\footrulewidth}{0.0pt}
% \footskip 63pt
% 
% %Tabellen
% 
% \usepackage{tabularx}
% \usepackage{float}
% \floatplacement{figure}{H}
% 


\subject{Analysis II}

\begin{document}

\begin{flashcard}[Metrischer Raum]{2. Abstandsmessung}

  \centerall{
    Wie ist der metrische Raum definiert?
}

\backside

\vspace*{1cm}

$\textbf{Modell } \text{metrischerRaum}$ \\
  $\textbf{benutzt } X,d \with$
  \begin{tabulator}
    $X : \text{Menge};$ \\
    $d : X \times X \rightarrow \R;$ \\
    $\forall x,y \with x,y \in X \then (d(x,y) = 0) \Leftrightarrow (x=y);$ \\
    $\forall x,y \with x,y \in X \then d(x,y) = d(y,x);$ \\
    $\forall x,y,z \with x,y,z \in X \then d(x,z) \leq d(x,y) + d(y,z);$
    \end{tabulator}
  $\Box$


\end{flashcard}




\begin{flashcard}[Wichtige Begriffe]{2. Abstandsmessung}

  \centerall{
   Definiere folgende Begriffe:
   \begin{itemize}
   \item[(a)] Grenzwert
   \item[(b)] Kugel
   \item[(c)] beschränkt
   \item[(d)] Cauchyfolge
   \end{itemize}
 }

 \backside

\begin{itemize}
\item[(a)] $\text{Grenzwert}(a \with a:\text{Folge}) := A \with A \in X; \forall \varepsilon
 \with \varepsilon > 0 \then \exists N \with N \in \N; \; \forall n \with
 \N_{\geq N} \then d(a_{n}, A) < \varepsilon \; \Box$

\item[(b)] $B(r,x \with r \in \R_{>0}; \; x \in X) := \{y \with d(y,x) < r\};$

\item[(c)] $\text{beschränkt} := M \with M \subseteq X; \; \exists R,x \with M
  \subseteq B_{R}(x) \; \Box \; \Box;$
\item[(d)] $\text{Cauchyfolge} := a \with a:\text{Folge}; \; \forall \varepsilon
  \with \epsilon > 0 \then \exists N \with N \in \N; \; \forall
  n,m \in \N_{\geq N} \then d(a_{n}, a_{m}) < \varepsilon \; \Box \; \Box;$
\end{itemize}
\end{flashcard}



\begin{flashcard}[Diskrete Metrik]{2. Abstandsmessung}

\centerall{
  Definiere diskrete Metrik.
  Welche Mengen sind mit der diskreten Metrik zusammen ein Generator für einen
  metrischen Raum?
}


  \backside

  $\text{diskreteMetrik} := x,y \with x,y \in X \rightarrow
    \begin{cases}
    1 & x \neq y \\
    0 & \text{sonst}
  \end{cases}
$

$\forall X \with X : \text{Menge} \then (X, \text{diskreteMetrik}(X)) : \text{Gen}(\text{metrischerRaum});$
\end{flashcard}


\begin{flashcard}[Vollständige Räume]{2. Abstandsmessung}

\begin{center}
  Sind i.A. alle Cauchyfolgen konvergent?\\
  Wann ist ein Raum vollständig?\\
  Gebe ein Beispiel für nicht vollständige Räume an. Wieso ist dieser nicht vollständig?
\end{center}


  \backside

  Im allgemeinen sind Cauchyfolgen \textbf{nicht} konvergent.\\
  $\text{vollständig} := M \with M:\text{metrischerRaum}; \; \forall a \with
  a:M.\text{Cauchyfolge} \then a:M.\text{konvergent} \; \Box$
  $M\R^{*} := metrischerRaum(\R^{*}, A1.d|_{\R^{*} \times \R^{*}})$
  Ist nicht vollständig, da $a := (\frac{1}{n})_{n \in \N}$ Cauchyfolge und divergent in
  $M\R^{*}$ ist.
  \end{flashcard}



\begin{flashcard}[Normierte Räume]{2. Abstandsmessung}


  \centerall{
    Wie ist der normierte Raum definiert?\\
    Was ist ein Banachraum?
}

\backside

$\textbf{Modell } \text{normierterRaum}$ \\
  $\textbf{benutzt } V,\text{Norm} \with$
  \begin{tabulator}
    $V : \text{Vektorraum};$ \\
    $V.K = \text{A1.reellerKörper}$; \\
    $X := V.X$; \\
    $0_{V} := V.0$; \\
    $\text{Norm} : X \rightarrow \R;$ \\
    $\forall x \with ||x|| = 0 \then x = 0_{V};$ \\
    $\forall \lambda,x \with \lambda \in \R; \; x \in X \then ||\lambda \odot
    x|| = |\lambda| \cdot ||x||;$ \\
    $\forall x,y \with x,y \in X \then ||x \oplus y|| \leq ||x|| + ||y||;$
    \end{tabulator}
  $\Box$

  $\text{Banachraum := R \with R : normierterRaum; \; R.M : \text{vollständig}}
  \; \Box;$


\end{flashcard}


\begin{flashcard}[]{2. Abstandsmessung}
  \centerall{
    Wie ist der Skalarproduktraum definiert?\\
    Was ist ein Hilbertraum?

}

\backside

$\textbf{Modell } \text{Skalarproduktraum}$ \\
  $\textbf{benutzt } V,\text{SP} \with$
  \begin{tabulator}
    $V : \text{Vektorraum};$ \\
    $V.K = \text{A1.reellerKörper}$; \\
    $X := V.X$; \\
    $0_{V} := V.0$; \\
    $\text{SP} : X \rightarrow \R;$ \\
    $\forall x \with x \in X \then \langle x,x \rangle \geq 0$; \\
    $\forall x \with \langle x,x \rangle = 0 \then x = 0_{V};$ \\
    $\forall x,y \with x,y \in X \then \langle x,y \rangle = \langle y,x \rangle;$ \\
    $\forall \lambda,x,y,z \with \lambda \in \R; \; x,y,z \in X \then \langle
    x,(\lambda \odot y) \oplus z \rangle = \lambda \cdot \langle x,y \rangle +
    \langle x,z \rangle;$
    \end{tabulator}
  $\Box$

  $\text{Hilertraum := H \with H : Skalarproduktraum; \; H.M : \text{vollständig}}
  \; \Box;$


\end{flashcard}



\begin{flashcard}[Winkel messen]{2. Abstandsmessung}


  \centerall{
    In welchem Raum können Winkel gemessen werden?\\
    Wie hängen Winkel und Skalarprodukt zusammen?\\
    Wie ist die Projektion definiert?\\
    Wie induziert der Skalarproduktraum eine Norm?\\
    Wie lautet die Cauchy-Schwarz-Ungleichung

  }


  \backside


  In Skalarproduktraum.\\

  $\cos \sphericalangle (x,y) = \frac{\langle x,y \rangle}{\sqrt{\langle x,x
      \rangle \langle y,y \rangle}}$\\

  $P(y \with y \in X; \; y \neq 0_{V}) := x \with x \in X \mapsto \frac{\langle
    x,y \rangle}{\langle y,y \rangle} \odot y;$\\

  $\text{Norm}(v \with v \in X) := \sqrt{\langle v,v \rangle};$\\

  $\forall x,y \with x,y \in X \then |\langle x,y \rangle| \leq ||x|| \cdot ||y||;$

\end{flashcard}


\begin{flashcard}[Beispiele]{2. Abstandsmessung}

  \centerall{

   Definiere StandardSkalarprodukt, p-, Unendlich-, Integral- und Sup-Norm.

  }

  \backside

  $\text{StandardSkalarprodukt}(x,y \with x,y : \text{diskretReellwertig}; \; \text{Def}(x) = \text{Def}(y)) := \sum \limits_{i \in \text{Def}(x)} x_{i} \cdot y_{i}$

  $\text{Potenznorm}(p \with p \in \R_{\geq 1}) := x \with x : \text{diskretReellwertig} \mapsto \left( \sum \limits_{i \in \text{Def(x}} |x_{i}|^{p} \right)^{\frac{1}{p}}$\\

  $\text{UnendlichNorm}(x \with x : \text{diskretReellwertig}) :=
  \begin{cases}
    \max \limits_{i \in \text{Def}(x)} |x_{i}| & \text{Def}(x) \neq \emptyset \\ 0 & \text{sonst}
  \end{cases}$

  $\text{Integralnorm}(p,a,b \with p \in \R_{\geq 1}; \; a < b) := f \with f \in \mathcal{F}(\I \left[a,b\right], \R); \; f: \text{stetig} \mapsto \left( \int \limits_{a}^{b} |f(x)|^{p} \text{d}x \right)^{\frac{1}{p}}$ Abk. $||f||_{\L^{p} (\I \left[ a,b \right])}$

  $\text{SupNorm}(f \with f : \text{Funktion}; \; \text{Bild}(f) \subseteq \R) :=
  \begin{cases}
    \sup \limits_{x \in \text{Def}(f)} |f(x)| & \text{Def}(f) \neq \emptyset \\
    0 & \text{sonst}
  \end{cases}$ Abk. $||f||_{\infty}$

\end{flashcard}


\begin{flashcard}[]{2. Abstandsmessung}



\centerall{

  Wann ist eine Folge in $\mathcal{F}(E,\R)$ konvergent?

}


\backside


\begin{satz}

Sei $N := \text{NormierterRaum}(\mathcal{F} (E,\R), ||\cdot||_{p})$ mit einer endlichen Menge $E$ und $p \in \overline{\R}_{\geq 1}$. Ist $a$ eine Folge mit Werten in $\mathcal{F}(E,\R)$, dann ist $a$ genau dann $N$.konvergent, wenn für alle $i \in E$ die Folge $(a_{n}(i))_{n \in \N}$ A1.konvergent sind. In diesem Fall gilt $N.\text{Limes}(a)(i) = \text{A1.Limes}((a_{n}(i))_{n \in \N})$ für jeder $i \in E$. Außerdem ist $N$ ein Banachraum.

\end{satz}

\end{flashcard}

\begin{flashcard}[]{2. Abstandsmessung}


  \centerall{

    Wie ist gleichmäßigKonvergent und punktweiseKonvergent definiert?

}


\backside

$\text{gleichmäßigKonvergentGegen} (A \with  A:\text{reellwertig}) := a \with a:\N \rightarrow \mathcal{F}(\text{Def}(A), \R); \; (||a_{n} - A||_{\infty})_{n \in \N} : \text{A1.Nullfolge} \; \Box;$\\
$\text{punktweiseKonvergentGegen}(A \with A:\text{reellwertig}) := a \with a:\N \rightarrow \mathcal{F}(\text{Def}, \R); \; \forall x \with x \in \text{Def}(A) \then (|a_{n}(x) - A(x)|)_{n \in \N} : \text{A1.Nullfolge} \; \Box;$

\backside

\end{flashcard}


\begin{flashcard}[Mengen im metrischen Raum]{3. Stetigkeit}

\centerall{
Definiere die Begriffe Umgebung, offen und abgeschlossen.
}

\backside

$\text{Umgebung}(x \with x \in X) := U \with U \subseteq X; \; \exists \varepsilon \with B_{\varepsilon}(x) \subseteq U \; \Box \Box ;$\\

$\text{offen} := U \with U \subseteq X; \; \forall x \with x \in U \then U : \text{Umgebung}(x) \; \Box ;$\\

$\text{abgeschlossen} := A \with A \subseteq X; \; X \setminus A : \text{offen} \; \Box ;$\\

$\forall x, r \with x \in X; \; r > 0 \then B_{r}(x) : \text{offen};$\\

$\forall x, U \with U \subseteq X; \; x \in X \then U : \text{Umgebung}(x) \Leftrightarrow \exists O \with x \in O; \; O \subseteq U; \; O : \text{offen} \; \Box$\\

\end{flashcard}



\begin{flashcard}[]{3. Stetigkeit}

\centerall{
  Definiere die Begriffe stetigIn, stetig und gebe die dazugehörigen Charakterisierungen.
  }

  \backside

$\text{stetigIn}(x, N, M \with N, M : \text{metrischerRaum}):= f \with D := \text{Def}(f); \; D \subset N.X; \; \text{Bild}(f) \subset M.X; \; x \in D; \; \forall a \with a : N. \text{Folge}; \; \text{Bild}(a) \subset D; \; N.\lim a = x \then M.\lim \limits_{n \to \infty} f(a_{n}) = f(x) \; \Box;$\\

$\text{stetig}(N, M \with N, M : \text{metrischerRaum}):= f \with f : N.X \mapsto M.X; \; \forall x \with x \in N.X \then f : \text{stetigIn}(x, N, M) \; \Box;$\\

$\forall f, x, N, M \with N, M : \text{metrischerRaum} \then f : \text{stetig}(x, N, M ) \Leftrightarrow \forall \varepsilon \with \varepsilon > 0 \then \exists \delta \with \delta > 0; \; \forall y \with y \in N.B_{\delta}(x) \then f(y) \in M.B_{\varepsilon}(f(x)) \; \Box;$\\

\newpage

$\forall f, N, M, x \with N, M : \text{metrischerRaum}; \; f : N.X \mapsto M.X; \; x \in N.X \then f : \text{stetigIn}(x, N, M ) \Leftrightarrow \forall U \with U : M. \text{Umgebung}(f(x)) \then f^{-1}(U) : N. \text{Umgebung}(x);$\\

$\forall f, N, M \with N, M : \text{metrischerRaum}; f : N.X \rightarrow M.X \then f \in C^{0}(N, M) \Leftrightarrow \forall U \with U : M. \text{offen} \then f^{-1}(U) : N. \text{offen};$\\

$\forall f, N, M \with N, M : \text{metrischerRaum}; f : N.X \rightarrow M.X \then f \in C^{0}(N, M) \Leftrightarrow \forall U \with U : M. \text{abgeschlossen} \then f^{-1}(U) : N. \text{abgeschlossen};$

\end{flashcard}


\begin{flashcard}[]{3. Stetigkeit}

  \centerall{
    Wie ist der topologische Raum definiert?
  }

\backside

$\textbf{Modell } \text{topologischerRaum}$ \\
  $\textbf{benutzt } X,\tau \with$
  \begin{tabulator}
    $X: \text{Menge};$ \\
    $\tau \subseteq \text{Pot}(X)$; \\
    $\text{offen} := U \with U \in \tau \; \Box $; \\
    $\emptyset : \text{offen}$; \\
    $X : \text{offen};$ \\
    $\forall \mathcal{M} \with \subseteq \tau; \; \mathcal{M} : \text{endlich} \then \bigcap \mathcal{M} : \text{offen};$ \\
    $\forall \mathcal{M} \with \subseteq \tau \then \bigcap \mathcal{M} : \text{offen};$ \\
    \end{tabulator}
  $\Box$

\end{flashcard}

\begin{flashcard}[]{3. Stetigkeit}

  $\text{Umgebung}(x \with x \in X) := U \with U \subseteq X; \exists O \with x \in O; \; O \subseteq U; \; O : \text{offen}$\\

  $\text{innererPunkt}(M \with M \subseteq X) := x \with M : \text{Umgebung}(x) ;$\\

  $\text{Inneres}(M \with M \subseteq X) := \{x \with M : \text{Umgebung}(x)\};$\\

  $\text{Häufungspunkt}(M \with M \subseteq X) := x \with x \in X; \; \forall U \with U : \text{Umgebung}(x) \then \exists y \with y \in M \bigcap U; \; y \neq x;$\\

  $\text{isolierterPunkt}(M \with M \subseteq X) := x \with x \in M; \neg (x : \text{Häufungspunkt}(M))$\\

$\text{Berührpunkt}(M \with M \subseteq X) := x \with x \in X; \; \forall U \with U : \text{Umgebung}(x) \then \exists y \with y \in M \bigcap U \; \Box \Box;$\\

$\text{Abschluss}(M \with M \subseteq X) := {x \with x :  \text{Berührpunkt}(M)};$\\

$\text{Randpunkt}(M \with M \subseteq X) := x \with x : \text{Berührpunkt}(M); \; x : \text{Berührpunkt}(X \setminus M);$\\

$\text{Rand}(M \with M \subseteq X) := {x \with x : \text{Randpunkt}(M)};$

\end{flashcard}

\end{document}
